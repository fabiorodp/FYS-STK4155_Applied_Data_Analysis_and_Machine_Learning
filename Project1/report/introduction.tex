\section{Introduction}

\quad \, Linear regression analysis performs a significant purpose in statistical modelling and is broadly applied nowadays. This technique is a linear approach to modelling the relationship between a response or dependent variable and explanatory or independent variables. In the case of only one explanatory variable, this is called a single linear regression, and for more than 2, we have a denominated multiple linear regression.\\

These relationships between variables are modelled using linear predictor functions, in other words, combinations of a set of unknown coefficients and known explanatory variables resulting in the response variables. Once we have a bunch of data observations, this is easy to infer the unknown coefficients from the techniques, and then we will end up with a model which can predict a response from any explanatory data.\\

The following topics of this dissertation will cover these techniques behind the linear regression, for instance, ordinary least squares, singular-value decomposition (SVD) and Ridge and Lasso regressions. Also, we will discuss how to access the model's accuracy by using r2 score, mean squared error and confidence interval of the regression's coefficients. Besides, we will additionally present the relevant machine learning topics, essentially cross-validation, re-sampling and scaling techniques.\\

Finally, after the consolidation of the theory, we will address and evaluate a specific real case of digital terrain data and apply the subjects studied.\\
