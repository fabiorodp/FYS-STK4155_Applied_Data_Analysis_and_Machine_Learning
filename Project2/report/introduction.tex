\section{Introduction}
\label{chap:Introduction}

\quad \, Supervised machine learning techniques for regression and classification scenarios have gained consistent developments throughout the last decade. Many different private and academic contributions have helped create more reliable, efficient, and precise methods to acquire equally or superior significance as the old traditional systems.\\

Artificial Neural Networks (ANN) techniques such that Multi-layer Perceptron (MLP) have proved an incredible potential. However, these models need to be studied and perfected even more by professionals and researchers. Our job here is to develop, analyze, and investigate different generalized techniques applied in as many different real domains as imaginable, with as much efficiency and accuracy as possible. This duty is the motivation of our scientific report, which will explore the subject of Neural Networks and see how this method can be enhanced by assessing more reliable results.\\

For that, this report will first investigate GeoTIF image data of an area near Stavanger in Norway on Gradient Descent (GD) models, essentially Momentum Gradient Descent (GDM), Stochastic Gradient Descent (SGD), Mini-batch Stochastic Gradient Descent (Mini-SGD), Mini-batch Momentum Stochastic Gradient Descent (Mini-SGDM). After that, we will apply Feed-Forward and Back-Propagation techniques in a model so-called Multi-Layer Perceptron (MLP) that fitts as a base for Neural Networks (NN) or Deep Neural Networks (DNN) principles. Therefore, we will examine the prediction accuracy results of its heights as a function of the model's parameters.\\

After that, we will cover a classification case called MNIST, which classifies handwritten digit images into one of its ten classes (0–9). The problem we are dealing with is essentially a Multi-class classification that we will be using the MLP model with selected activation and cost functions.\\

In the end, we will analyze the breast cancer data, a binary classification problem with response variable 0 for patients not containing breast cancer and 1 for breast cancer patients. Our MLP model with selected activation and cost functions will also be applied to this problem. We will likewise apply a binary classification model called Logistic Regression on this data-set and then compare it to the MLP model.\\

The following topics of this dissertation will cover these methods under the studied models. First, this report will approach the \hyperref[chap:Theory]{theories} utilized in the \hyperref[chap:Discussion]{discussion section}. After, it will divide the discussion and result topics into different subsections, each one dealing with a different subject and with a separate python test file in our GitHub repository. In the end, a \hyperref[chap:Conclusion]{conclusion} will discuss the pros and cons of the methods and possible improvements and perspectives for future work.

\subsection{Source code}
\label{chap:Source code}

\quad \, All codes utilized in this project is written in Python v.3.8, and found in the GitHub repository at \href{https://github.com/fabiorodp/UiO-FYS-STK4155/tree/master/Project2}{https://github.com/fabiorodp/UiO-FYS-STK4155/tree/master/Project2}. The repository contains:

\begin{itemize}
\item \href{https://github.com/fabiorodp/UiO-FYS-STK4155/tree/master/Project2/data/SRTM_data_Norway_1.tif}{data/SRTM\_data\_Norway\_1.tif} - Data utilized in this project.
\item \href{https://github.com/fabiorodp/UiO-FYS-STK4155/tree/master/Project2/package/}{package/} - Directory containing the collection of python's methods utilized in the project.
\item \href{https://github.com/fabiorodp/UiO-FYS-STK4155/tree/master/Project2/report/}{report/} - Directory containing the written report in latex and pdf.
\item \href{https://github.com/fabiorodp/UiO-FYS-STK4155/tree/master/Project2/partA.py}{partA.py} - Test script for part A of the project.
\item \href{https://github.com/fabiorodp/UiO-FYS-STK4155/tree/master/Project2/partBandC.py}{partBandC.py} - Test script for part B and C of the project.
\item \href{https://github.com/fabiorodp/UiO-FYS-STK4155/tree/master/Project2/partD.py}{partD.py} - Test script for part D of the project.
\item \href{https://github.com/fabiorodp/UiO-FYS-STK4155/tree/master/Project2/partE.py}{partE.py} - Test script for part E of the project.
\end{itemize}
