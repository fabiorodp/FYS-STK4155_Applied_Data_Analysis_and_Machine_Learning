\section{Conclusions and Future Work}
\label{chap:Conclusions and Future Work}

\quad This project introduces a day-trading algorithm tool for the decision-making process that uses the outputs from RNN and LSTM. The ANNs sub-classes predict the next day's highest and lowest prices as outputs and then use them as inputs for a day-trading algorithm system. The metrics used for assessing the predictions were MSE and MAE values. It is essential to say that the data's scaling is not applied because we want to visualize the numbers (as it is) and check possible improvements. Regarding our suggested trading system, it uses specific rules as triggers for entering and exiting the market. Back-tests are performed on PETR4 stock data from B3 Sao Paulo's stock exchange, which has the highest liquidity flow (financial volume) in the Brazilian market. Results are compared with a traditional trading system using a TA instrument called Bollinger Bands, our bench-mark.\\

Our trading system's best result (BRL10.76 per share of cumulative profit) operated approx. five times as better as the bench-mark (BRL1.95 per share of cumulative profit), confirming that ANN usage, especially the RNN, has outstanding profitability and great advance against traditional techniques. Indeed, the targets' validated predictions get accuracy metrics as low as 0.5, which are very good, using both RNN and LSTM. Although LSTM is extremely expensive for training computations and might not be adequate for high-speed markets nowadays.\\

As a future research objective, it would be relevant to explore other ANN techniques like Gated Recurrent Unit (GRU), as well as many additional TA indicators as features for training the ANN. Besides, scaling data, a larger number of units, hidden-layers, epochs, and mini-batch could be considered in the future, but it will significantly increase the computation costs avoided in this project. Moreover, the quantity of historical data could also be examined to have a long-term image of performance and profits. Other securities could similarly be experimented with for checking the strategy's feasibility in different types of stocks, markets, countries, and others. Finally, trading policies for risk management and other tools for avoiding order issues like slippages could be analyzed in future works as additional protection against turnovers or unexpected market situations.
