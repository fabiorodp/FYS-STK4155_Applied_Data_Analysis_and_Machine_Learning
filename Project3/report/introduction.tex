\section{Introduction}
\label{chap:Introduction}

\quad \, This scientific report's idea consists of approaches in the area of Quantitative Finance Analysis field using computational mathematical models directly inspired by the human brain structure, so-called  Artificial Neural Network (ANN) methods, for instance, Neural Network (NN) and Recurrent Neural Network (RNN).\\

Nowadays, professionals and quantitative researchers rely on many traditional and contemporary computational strategies for decision-making based on speed and accuracy. Therefore, in the face of big data analytic advances, the challenge is to find a reliable tool suitable for interpreting large amounts of different unstructured data available in the market today. Indeed, the dynamics, complexities, evolutive, and chaotic nature of the markets make this task unmanageable and sometimes ineffective.\\

For that, this report will explore computational learning process techniques to generalize the information obtained from interactions with the financial market environment. Today uncountable many promising researchers of ANN techniques highlight their capability to resolve complex sequential decision-making dilemmas, and this should not be a problem when applied in the financial time series forecasting field.\\

The assignment starts collecting and handling the various available data from B3 Sao Paulo's Stock Exchange \hyperref[Bib:b3 quotes]{[8]}, filtering, creating, and labeling the best features.  In parallel, RNN and LSTM theories desire to be included and addressed. Next, it will be necessary to find feasible ways of feeding the ANN systems with the processed data. Then, assessment of the results, adjustments, new inquiries, and ideas could be analyzed. In the end, the research aims to propose a application of the optimized models to allow experts to enhance returns, reduce risk, and increase efficiency by systematically incorporating ANN outcomes for algorithm trading predictions.

\subsection{Source code}
\label{chap:Source code}

\quad \, All codes utilized in this project is written in Python v.3.8, and found in the GitHub repository at \href{https://github.com/fabiorodp/UiO-FYS-STK4155/tree/master/Project3}{https://github.com/fabiorodp/UiO-FYS-STK4155/tree/master/Project3}. The repository contains:

\begin{itemize}
\item \href{https://github.com/fabiorodp/UiO-FYS-STK4155/tree/master/Project2/data/SRTM_data_Norway_1.tif}{data/SRTM\_data\_Norway\_1.tif} - Data utilized in this project.
\item \href{https://github.com/fabiorodp/UiO-FYS-STK4155/tree/master/Project2/package/}{package/} - Directory containing the collection of python's methods utilized in the project.
\item \href{https://github.com/fabiorodp/UiO-FYS-STK4155/tree/master/Project2/report/}{report/} - Directory containing the written report in latex and pdf.
\item \href{https://github.com/fabiorodp/UiO-FYS-STK4155/tree/master/Project2/partA.py}{partA.py} - Test script for part A of the project.
\item \href{https://github.com/fabiorodp/UiO-FYS-STK4155/tree/master/Project2/partBandC.py}{partBandC.py} - Test script for part B and C of the project.
\item \href{https://github.com/fabiorodp/UiO-FYS-STK4155/tree/master/Project2/partD.py}{partD.py} - Test script for part D of the project.
\item \href{https://github.com/fabiorodp/UiO-FYS-STK4155/tree/master/Project2/partE.py}{partE.py} - Test script for part E of the project.
\end{itemize}
