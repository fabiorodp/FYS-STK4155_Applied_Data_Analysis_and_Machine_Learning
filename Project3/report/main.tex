\documentclass{article}
\usepackage[utf8]{inputenc}
\usepackage{amsmath}
\usepackage{listings}
\usepackage{hyperref}
\usepackage{graphicx}
\usepackage{float}
\usepackage{import}
\usepackage{tabularx}
\usepackage{multirow}
\usepackage[table]{xcolor}
\usepackage[a4paper, total={6in, 9in}]{geometry}
\graphicspath{ {./pictures/} }

\usepackage{makeidx}
\makeindex

\title{Project 3: UiO: FYS-STK4155: Financial time series forecasting using Recurrent Neural Networks (RNN) and Long Short-Term Memory Cells (LSTM) and application}
\author{\textbf{Fábio Rodrigues Pereira} \\ \small fabior@uio.no - github: @fabiorod}
\date{December 16, 2020}

\begin{document}
\maketitle
\begin{abstract}
\noindent Predicting the stock market always is a topic of significant interest for scholars and professionals. However, the chaotic dynamics of the markets make this kind of study complex and challenging. In order to overcome these obstacles, this project proposes a day-trading tool based on outputs of Recurrent Neural Networks (RNN) and Long-Short Term Memory Networks (LSTM) for the decision-making process. We perform an RNN and an LSTM, with various combinations of parameters, to predict the next day's highest and lowest prices for the stock PETR4 (B3 Sao Paulo's Stock Exchange) and employ the best-trained network to an algorithm trading problem. Our best scenario, under RNN model, achieved training MSE accuracy of 0.38 and 0.36, Cross-Validated testing MSE of 0.48 and 0.45, training MAE of 0.43 and 0.40, and CV testing MAE of 0.45 and 0.50, for predicting the next day's highest and lowest, respectively. Besides, we compare our best network's performance in real-data with a traditional trading strategy (5-day Bollinger Bands) bench-marked. From 14.10.200 to 03.12.2020, based on our best RNN, our day-trading strategy makes a promising cumulative daily return of BRL10,76 per share (cumulative daily percentage of 39\%), overcoming the traditional strategy that achieved a cumulative daily return of BRL1,95 per share (cumulative daily percentage of 7\%).
\end{abstract}

\clearpage
\thispagestyle{empty}

\tableofcontents

\clearpage
\thispagestyle{empty}

\import{./}{introduction.tex}

\clearpage
\thispagestyle{empty}

\import{./}{theory.tex}

\clearpage
\thispagestyle{empty}

\import{./}{discussion.tex}

\clearpage
\thispagestyle{empty}

\import{./}{conclusion.tex}

\clearpage
\thispagestyle{empty}

\import{./}{bibliography.tex}

\end{document}
